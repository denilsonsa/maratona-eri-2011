\problema{Loteria falha}

\begin{descricao}
Muita gente sonha em ganhar dinheiro f�cil. Algumas pessoas tentam fazer isso atrav�s da loteria. Compram jogos como as ``raspadinhas'' e bilhetes de loteria aguardando sorteios multimilion�rios.

Gilmar, um rapaz muito malandro, decidiu usar seus conhecimentos matem�ticos para tentar aumentar suas chances nestes jogos de sorte. Ele comprou v�rios bilhetes de um mesmo tipo de raspadinha e analisou as cartelas, at� que percebeu uma propriedade muito curiosa: cada raspadinha tinha impresso um n�mero identificador do cart�o que permitia a ele ter no��o das chances de ser premiado.

De cada $10$ cart�es que comprou na banca, aproximadamente $1$ ou $2$ vinham premiados de alguma forma: no m�nimo uma outra raspadinha gr�tis ele ganhava, ou um pr�mio simb�lico em dinheiro. Quando aplicou seu m�todo para escolher que cart�es comprar, percebeu que de cada $10$ cart�es, em m�dia $8$ continham algum pr�mio!

Como a tarefa � muito cansativa para ser feita manualmente, ele pensou que voc�, amigo de longa data, poderia ajud�-lo com um programa que, dado o n�mero identificador do cart�o, diz se ele faz parte do conjunto de cart�es com maior chance de pr�mio. O truque � procurar os cart�es cujo n�mero identificador seja m�ltiplo de $42$. Voc� est� apto a ajudar seu colega nesta empreitada?
\end{descricao}

\begin{entrada}
A entrada � composta por v�rios casos de teste, um por linha.

Cada caso de teste cont�m um inteiro $n$ de at� $30$ d�gitos decimais.

A entrada termina com $n = 0$. Este caso n�o dever� ser processado.\end{entrada}

\begin{saida}
Para cada caso de teste haver� uma linha na sa�da. Caso o n�mero identificador do cart�o seja um m�ltiplo de $42$, imprima ``PREMIADO''. Caso contr�rio, imprima ``TENTE NOVAMENTE''.
\end{saida}

\exemplos{0.47}{0.47}
{42\pl
17283940536172938433\pl
17283940536172938432\pl
10000000000000000000\pl
0
}
{PREMIADO\pl
TENTE NOVAMENTE\pl
PREMIADO\pl
TENTE NOVAMENTE
}

%\pb
