\problema{Os �ltimos ser�o os primeiros}

\begin{descricao}
�ric Ruiz Irrigado, o famoso Er�, � conhecido entre seus amigos por querer fazer previs�es. Em todo tipo de competi��o ou evento esportivo ele sempre tenta adivinhar os vencedores, os perdedores, artilheiros e coisas similares. Apesar das brincadeiras e deboches de seus amigos, Er� nunca desistiu e sempre busca padr�es onde os outros v�em apenas coincid�ncias.

Acompanhando os times da Maratona de Programa��o, Er� percebeu que a coloca��o dos times de seu estado na primeira fase sempre se invertiam na segunda fase, ainda que outros times de outras regi�es do pa�s estivessem entre eles. Assim, se o time da \emph{Uni1} ficar na frente da \emph{Uni2} na primeira fase, Er� imagina que o time da \emph{Uni2} ficar� na frente do time da \emph{Uni1} na segunda fase.

Para validar sua hip�tese, ele quer desenvolver um programa que, dada uma lista de coloca��o dos times na primeira fase, mostre qual ser� a posi��o relativa destes mesmos times na segunda fase.
\end{descricao}

\begin{entrada}
A entrada � composta por diferentes casos de teste. A primeira linha de cada caso de teste cont�m $n \leq 100$, o n�mero de times do estado de Er�. As $n$ linhas seguintes conter�o n inteiros distintos entre $1$ e $n$, inclusive, um por linha, cada inteiro representando um time.

A entrada termina com $n=0$. Essa linha n�o deve ser processada.

\end{entrada}

\begin{saida}
Para cada caso de teste, seu programa deve imprimir a posi��o relativa de cada um dos times de acordo com a previs�o de Er�, com um n�mero por linha. Ap�s a lista de times, deve ser impressa uma linha contendo um �nico ``0''. Veja o exemplo abaixo.


\end{saida}

\exemplos{0.47}{0.47}
{5\pl
3\pl
4\pl
2\pl
5\pl
1\pl
2\pl
1\pl
2\pl
0
}
{1\pl
5\pl
2\pl
4\pl
3\pl
0\pl
2\pl
1\pl
0
}

%\pb
