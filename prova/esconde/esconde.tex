\problema{Pique Esconde}

\begin{descricao}
Jo�o, que gosta de inform�tica, procura pensar em como seria um programa para automatizar tudo que ele faz no seu dia a dia. Maria o desafiou a fazer um programa que fosse �til de algum jeito para quando eles brincarem de pique esconde.

Jo�o teve uma ideia muito boa de um programa, mas como ele ainda est� em suas primeiras aulas de inform�tica, ainda n�o descobriu como implementar essa ideia. Assim, voc� deve ajud�-lo. A ideia de Jo�o � numerar as crian�as participando da brincadeira de $1$ a $n$. O programa recebe como entrada o n�mero total de crian�as e tamb�m o n�mero correspondente � crian�a que est� procurando os demais participantes. Depois, o programa recebe uma lista com $n-2$ n�meros identificando as crian�as que j� foram encontradas e deve responder qual a crian�a que est� faltando.

Sua tarefa � implementar esse programa e ajudar Jo�o a mostrar para Maria que a inform�tica pode ser �til at� nas mais simples tarefas.

\end{descricao}

\begin{entrada}
A entrada � composta por diferentes casos de teste. A primeira linha de cada caso de teste cont�m $n$, o n�mero de crian�as, e $m$, o n�mero da crian�a que est� procurando as demais, $1 \leq n \leq 1000$ e $1 \leq m \leq n$. As $n-2$ linhas seguintes cont�m um inteiro cada representando as crian�as que j� foram encontradas.

A entrada termina com $n=m=0$. Essa linha n�o deve ser processada.

\end{entrada}

\begin{saida}
Para cada caso de teste, seu programa deve imprimir uma linha contendo o n�mero da crian�a que ainda n�o foi encontrada.


\end{saida}

\exemplos{0.47}{0.47}
{3 2\pl
1\pl
5 1\pl
2\pl
3\pl
4\pl
4 4\pl
3\pl
2\pl
0 0 
}
{3\pl
5\pl
1
}

%\pb
