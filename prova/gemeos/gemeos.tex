\problema{Centros de Distribui��o G�meos}

\begin{descricao}
A Empresa Rural Internacional (ERI) est� buscando expandir seus neg�cios. Para isso, ela precisa encontrar uma nova cidade para instalar um novo centro de distribui��o. Entretanto, como algum centro pode ficar inoperante, como aconteceu no �ltimo ver�o devido �s chuvas, ela quer se certificar de que n�o ficar� sem distribuir seus produtos caso alguma cidade fique incomunic�vel. Para isso, ao inv�s de instalar seu novo centro de distribui��o em uma �nica cidade, ela escolher� duas cidades vizinhas, de forma a garantir que haja redund�ncia, caso algum problema ocorra com uma das localidades.

Simonal, diretor de log�stica da ERI, conhece bem a regi�o e seus poss�veis clientes. Dada uma cidade, Simonal sabe a �rea de alcance de sua empresa, ou seja, as cidades alcan�adas caso um novo centro de distribui��o seja instalado nessa cidade. A fim de que a redund�ncia desejada seja satisfeita, as duas cidades candidatas a receber o novo centro de distribui��o devem ter a mesma �rea de alcance.

Voc�, estagi�rio buscando um aumento de sal�rio para comprar a mais nova expans�o do seu MMORPG favorito, quer agradar o seu chefe, Dion�sio Cabe�o Fechantes, tamb�m conhecido como Dudu, e, usando os dados obtidos com Simonal, quer mostrar o qu�o bom voc� � em computa��o e contar de quantas formas distintas estas cidades podem ser escolhidas.\end{descricao}

\begin{entrada}
A entrada � composta por diferentes casos de teste. A primeira linha de cada caso de teste cont�m $n$ e $m$, $n\leq1000$ e $m \leq \frac{n(n-1)}{2}$, onde $n$ � o n�mero de cidades da regi�o. A seguir, $m$ linhas, contendo dois inteiros distintos $a$ e $b$ cada, $1 \leq a,b \leq n$, indicando que $a$ faz parte da �rea de alcance de $b$ e vice-versa.

A entrada termina com $n=m=0$. Essa linha n�o deve ser processada.

\end{entrada}



\begin{saida}
Para cada caso de teste, seu programa deve imprimir uma linha contendo o n�mero de diferentes pares de cidades que podem receber novos centros.
\end{saida}

\exemplos{0.47}{0.47}
{3 2\pl
1 2\pl
2 3\pl
3 3\pl
3 2\pl
1 3\pl
2 1\pl
5 5\pl
1 2\pl
1 5\pl
2 3\pl
2 5\pl
4 5\pl
0 0 
}
{0\pl
3\pl
0
}

%\pb
