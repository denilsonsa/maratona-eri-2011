\problema{Cada um no seu quadrado}

\begin{descricao}
Ana e Bob gostam de brincar de geometria. Semana passada, enquanto ouviam uma can��o que falava de quadrados, Bob, que estava brincando com seus blocos de madeira, se perguntou quantos quadrados poderiam ser formados usando os blocos como v�rtices, mas sem mov�-los. Ana, que aprendeu a programar recentemente, teve a ideia de desenvolver um programa para resolver esse problema. Entretanto, Ana teve dificuldade para resolver esse problema e decidiu pedir a sua ajuda. Como os blocos s�o pequenos quando comparados � dist�ncia entre eles, voc� pode assumir que eles s�o pontos em um plano.
\end{descricao}

\begin{entrada}
A entrada consiste de multiplos casos de teste. Cada caso de teste come�a com uma linha contendo um �nico inteiro $4 \leq n \leq 1.000$. As $n$ linhas seguintes cont�m $2$ inteiros $x$ e $y$ cada, $-1.000.000 \leq x, y \leq 1.000.000$, referentes �s coordenadas dos blocos. A entrada termina com uma linha contendo $n=0$, que n�o deve ser processada.\end{entrada}

\begin{saida}
Para cada caso de teste, voc� deve imprimir uma linha contendo um �nico inteiro, o n�mero de quadrados que podem ser obtidos a partir das posi��es dadas.
\end{saida}

\exemplos{0.47}{0.47}
{6\pl
1 0\pl
0 1\pl
2 1\pl
1 2\pl
0 -1\pl
2 -1 \pl
0
}
{2
}

%\pb
